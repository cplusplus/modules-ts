%!TEX root = std.tex
\setcounter{chapter}{13}
\rSec0[cpp]{Preprocessing directives}%

\begin{after}
Modify paragraph 14/5 as follows:

\begin{std.txt}
\pnum[5]
The implementation can
process and skip sections of source files conditionally,
include other source files,
\added{import macros from header units,}
and replace macros.
These capabilities are called
\term{preprocessing},
because conceptually they occur
before translation of the resulting translation unit.
\end{std.txt}
\end{after}

\setcounter{section}{0}
\rSec1[cpp.cont]{Conditional inclusion}%

Modify the grammar before 14.1/1 as follows:

\begin{std.txt}
\begin{bnf}
\added{\nonterminal{header-name-tokens}:}\br
  \added{string-literal}\br
  \added{\terminal{<} h-pp-tokens \terminal{>}}
\end{bnf}

\begin{bnf}
\nonterminal{has-include-expression}:\br
  % FIXME: Need \textunderscore: underscore package does not work inside \added / \removed!
  \removed{\terminal{\xname{has\textunderscore{}include}} \terminal{(} \terminal{<} h-char-sequence \terminal{>} \terminal{)}}\br
  \removed{\terminal{\xname{has\textunderscore{}include}} \terminal{(} \terminal{"} q-char-sequence \terminal{"} \terminal{)}}\br
  \added{\terminal{\xname{has\textunderscore{}include}} \terminal{(} header-name \terminal{)}}\br
  \removed{\terminal{\xname{has\textunderscore{}include}} \terminal{(} string-literal \terminal{)}}\br
  \removed{\terminal{\xname{has\textunderscore{}include}} \terminal{(} \terminal{<} h-pp-tokens \terminal{>} \terminal{)}}\br
  \added{\terminal{\xname{has\textunderscore{}include}} \terminal{(} header-name-tokens \terminal{)}}\br
\end{bnf}
\end{std.txt}

Modify paragraph 14.1/2 as follows:

\begin{std.txt}
\pnum[2]
A \grammarterm{defined-macro-expression} evaluates to \tcode{1}
if the identifier is currently defined as a macro name
(that is, if it is predefined or if
\added{it has one or more active macro definitions (\ref{cpp.module}),
for example because}
it has been the subject of a \tcode{\#define} preprocessing directive
without an intervening \tcode{\#undef} directive with the same subject
identifier),
\tcode{0} if it is not.

\end{std.txt}

Modify paragraph 14.1/3 as follows:

\begin{std.txt}
\pnum[3]
The \removed{third and fourth forms}
\added{second form}
of \grammarterm{has-include-expression}
\removed{are} \added{is} considered
only if \removed{neither of} the first \removed{or second forms matches}
\added{form does not match},
in which case the preprocessing tokens are processed just as in normal text.
\end{std.txt}

\setcounter{section}{1}
\rSec1[cpp.include]{Source file inclusion}%

\begin{after}
Add a new paragraph after 14.2/6 as follows:

\begin{std.txt}
\color{addclr}
\pnum[7]
If the header identified by the \grammarterm{header-name}
denotes an importable header (\ref{module.import}),
the preprocessing directive
is instead replaced by the \grammarterm{preprocessing-token}{s}

\begin{bnf}
\terminal{import} header-name \terminal{;}
\end{bnf}
\end{std.txt}
\end{after}

\noindent
Add a new subclause 14.3 titled ``\textbf{Global module fragment}'' as follows:

\setcounter{section}{2}
\rSec1[cpp.glob.frag]{Global module fragment}%

\begin{std.txt}
\color{addclr}
\begin{bnf}
\nonterminal{pp-global-module-fragment}:\br
  \terminal{module} \terminal{;} pp-bracketed-tokens \terminal{module}
\end{bnf}

\pnum
If the first two preprocessing tokens at the start of phase 4 of translation
are \tcode{module} \tcode{;}, the result of preprocessing shall begin with
a \grammarterm{pp-global-module-fragment} for which all
\grammarterm{preprocessing-token}{s} in the \grammarterm{pp-bracketed-tokens}
were produced directly or indirectly by source file inclusion
(\ref{cpp.include}), and for which the second \tcode{module}
\grammarterm{preprocessing-token} was not produced by source file inclusion or
macro replacement (\stdref{cpp.replace}{14.3}).
Otherwise, the first two preprocessing tokens at the end of phase 4 of
translation shall not be \tcode{module} \tcode{;}.
\end{std.txt}

Add a new subclause 14.4 titled ``\textbf{Legacy header units}'' as follows:

\setcounter{section}{3}
\rSec1[cpp.module]{Legacy header units}%

\begin{std.txt}
\color{addclr}
\begin{bnf}
\nonterminal{import-seq}:\br
  \grammarterm{top-level-token-seq}\opt{} \terminal{export}\opt{} \terminal{import}
\end{bnf}

\begin{bnf}
\nonterminal{top-level-token-seq}:\br
  \descr{any \nonterminal{pp-balanced-token-seq} ending in \terminal{;} or \terminal{\}}}
\end{bnf}

\begin{bnf}
\nonterminal{pp-import}:\br
  \terminal{import} header-name pp-import-suffix\opt{} \terminal{;}\br
  \terminal{import} header-name-tokens pp-import-suffix\opt{} \terminal{;}
\end{bnf}

\begin{bnf}
\nonterminal{pp-import-suffix}:\br
  pp-import-suffix-token\br
  pp-import-suffix pp-import-suffix-token
\end{bnf}

\begin{bnf}
\nonterminal{pp-import-suffix-token}:\br
  \descr{any \grammarterm{pp-balanced-token} other than \terminal{;}}
\end{bnf}

\begin{bnf}
\nonterminal{pp-balanced-token-seq}:\br
  pp-balanced-token\br
  pp-balanced-token-seq pp-balanced-token
\end{bnf}

\begin{bnf}
\nonterminal{pp-balanced-token}:\br
  pp-ldelim pp-balanced-token-seq\opt{} pp-rdelim\br
  \descr{any \grammarterm{preprocessing-token} other than a \nonterminal{pp-ldelim} or \nonterminal{pp-rdelim}}
\end{bnf}

\begin{bnf}
\nonterminal{pp-ldelim:} \descr{one of}\br
  \terminal{(    [    \{    <:    <\%}
\end{bnf}

\begin{bnf}
\nonterminal{pp-rdelim:} \descr{one of}\br
  \terminal{)    ]    \}    :>    \%>}
\end{bnf}

\color{addclr}
\pnum
A sequence of \grammarterm{preprocessing-token}{s} matching the form
of a \grammarterm{pp-import}
instructs the preprocessor to import macros from the header unit
(\ref{module.import}) denoted by the \grammarterm{header-name}.
A \grammarterm{pp-import} is only recognized when the sequence of tokens
produced by phase 4 of translation up to the \tcode{import} token
forms a \grammarterm{import-seq}, and the \tcode{import} token is not
within the \grammarterm{pp-import-suffix} of another \grammarterm{pp-import}.
The \tcode{;} \grammarterm{preprocessing-token} shall not be produced by
macro replacement (\stdref{cpp.replace}{14.3}).
The \term{point of macro import} for a \grammarterm{pp-import} is
immediately after the \tcode{;} terminating the \grammarterm{pp-import}.

\color{addclr}
\pnum
In the second form of \grammarterm{pp-import},
a \grammarterm{header-name} token is formed as if
the \grammarterm{header-name-tokens}
were the \grammarterm{pp-tokens} of a \tcode{\#include} directive.
The \grammarterm{header-name-tokens} are replaced by
the \grammarterm{header-name} token.
\begin{note}
This ensures that imports are treated consistently by
the preprocessor and later phases of translation.
\end{note}

\color{addclr}
\pnum
Each \tcode{\#define} directive encountered when preprocessing
each translation unit in a program results in a distinct
\term{macro definition}.
Importing macros from a header unit makes macro definitions
from a translation unit visible in other translation units.
Each macro definition has at most one point of definition in
each translation unit and at most one point of undefinition, as follows:
\begin{itemize}
\item
The \term{point of definition} of a macro definition within a translation unit
is the point at which its \tcode{\#define} directive occurs (in the translation
unit containing the \tcode{\#define} directive), or,
if the macro name is not lexically identical to a keyword (\stdref{lex.key}{5.11})
or to the \grammarterm{identifier}{s} \tcode{module} or \tcode{import},
the first point
of macro import of a translation unit containing a point of definition for the
macro definition, if any (in any other translation unit).

\item
The \term{point of undefinition} of a macro definition within a translation unit
is the first point at which a \tcode{\#undef} directive naming the macro occurs
after its point of definition, or the first point
of macro import of a translation unit containing a point of undefinition for the
macro definition, whichever (if any) occurs first.
\end{itemize}

\pnum
A macro directive is \term{active} at a source location
if it has a point of definition in that translation unit preceding the location,
and does not have a point of undefinition in that translation unit preceding
the location.

\pnum
If a macro would be replaced or redefined, and multiple macro definitions
are active for that macro name, the active macro definitions shall all be
valid redefinitions of the same macro (\stdref{cpp.replace}{14.3}).
\enternote
The relative order of \grammarterm{pp-import}{s} has no bearing on whether a
particular macro definition is active.
\exitnote

\pnum
\begin{example}
\begin{codeblock}
// translation unit a.h
#define X 123 // \#1
#define Y 45  // \#2
#define Z a   // \#3
#undef X      // point of undefinition of \#1 in \tcode{a.h}
\end{codeblock}

\begin{codeblock}
// translation unit b.h
import "a.h"; // point of definition of \#1, \#2, and \#3, point of undefinition of \#1 in \tcode{b.h}
#define X 456 // OK, \#1 is not active
#define Y 6   // ill-formed; \#2 is active
\end{codeblock}

\begin{codeblock}
// translation unit c.h
#define Y 45  // \#4
#define Z c   // \#5
\end{codeblock}

\begin{codeblock}
// translation unit d.h
import "a.h"; // point of definition of \#1, \#2, and \#3, point of undefinition of \#1 in \tcode{d.h}
import "c.h"; // point of definition of \#4 and \#5 in \tcode{d.h}
int a = Y;    // OK, active macro definitions \#2 and \#3 are valid redefinitions
int c = Z;    // ill-formed; active macro definitions \#2 and \#3 are not valid redefinitions of \tcode{Z}
\end{codeblock}
\end{example}
\end{std.txt}
