%!TEX root = std.tex
\setcounter{chapter}{18}
\rSec0[cpp]{Preprocessing directives}%

\begin{after}
Modify paragraph 19/5 as follows:

\begin{std.txt}
\resetalinea[4]
\alinea
The implementation can
process and skip sections of source files conditionally,
include other source files,
\added{import macros from legacy header units,}
and replace macros.
These capabilities are called
\term{preprocessing},
because conceptually they occur
before translation of the resulting translation unit.
\end{std.txt}
\end{after}

\setcounter{section}{1}
\rSec1[cpp.include]{Source file inclusion}%

\begin{after}
Add a new paragraph after 19.2/6 as follows:

\begin{std.txt}
\resetalinea[6]
\color{addclr}
\alinea
If the header identified by the \grammarterm{header-name}
denotes a legacy header unit, the preprocessing directive
is instead replaced by the \grammarterm{preprocessing-token}{s}
\begin{bnf}
\terminal{import} header-name \terminal{;}
\end{bnf}
\color{addclr}
How the set of headers denoting legacy header units is specified
is implementation-defined.
\end{std.txt}
\end{after}

\begin{after}
\noindent
Add a new subclause 19.3 titled ``\textbf{Legacy header units}'' as follows:

\setcounter{section}{2}
\rSec1[cpp.module]{Legacy header units}%
\resetalinea[0]

\begin{std.txt}
\color{addclr}
\begin{bnf}
\nonterminal{pp-import}:\br
  \terminal{export}\opt{} \terminal{import}\opt{} header-name pp-attrs\opt{} \terminal{;}
\end{bnf}

\begin{bnf}
\nonterminal{pp-attrs}:\br
  pp-attrs\opt{} \terminal{[} \terminal{[} pp-attr-tokens \terminal{]} \terminal{]}
\end{bnf}

\begin{bnf}
\nonterminal{pp-attr-tokens}:\br
  pp-attr-tokens\opt{} pp-attr-token
\end{bnf}

\begin{bnf}
\nonterminal{pp-attr-token}:\br
  \descr{any \grammarterm{preprocessing-token} other than \terminal{[} or \terminal{]}}
\end{bnf}

\begin{bnf}
\nonterminal{pp-preamble}:\br
  \terminal{module}\opt{} pp-module-name pp-attrs\opt{} \terminal{;}
  pp-preamble pp-import
\end{bnf}

\begin{bnf}
\nonterminal{pp-module-name}:\br
  pp-module-name\opt{} pp-module-name-token
\end{bnf}

\begin{bnf}
\nonterminal{pp-module-name-token}:\br
  \descr{any \grammarterm{preprocessing-token} other than \terminal{[} or \terminal{;}}
\end{bnf}

\color{addclr}
\alinea
A sequence of \grammarterm{preprocessing-token}{s} matching the form
of a \grammarterm{pp-import}
instructs the preprocessor to import macros from the legacy header unit
(\ref{dcl.module.import}) denoted by the \grammarterm{header-name}.
The \tcode{;} \grammarterm{preprocessing-token} shall not be produced by
macro replacement (\stdref{cpp.replace}{19.3}).
The \term{point of macro import} for a \grammarterm{pp-import} is
immediately after the \tcode{;} terminating the preprocessing preamble
(see below) if the import occurs within the preprocessing preamble, and
immediately after the \tcode{;} terminating the \grammarterm{pp-import}
otherwise.

\noindent % force para to begin so color applies to para number
\color{addclr}
\alinea
A \term{macro directive} for a macro name is a \tcode{\#define} or
\tcode{\#undef} directive naming that macro name.
An \term{exported macro directive} is
a macro directive occuring in a legacy header unit
whose macro name is not lexically identical to a keyword.
A macro directive is \term{visible} at a source location
if it precedes that source location in the same translation unit, or
if it is an exported macro directive whose legacy header unit,
or a legacy header unit that transitively imports it,
is imported into the current translation unit by a \grammarterm{pp-import}
whose point of macro import precedes that source location.

\alinea
Multiple macro directives for a macro name may be visible at the same
source location.
The interpretation of a macro name is determined as follows:
\begin{itemize}
\item
\color{addclr}
A macro directive \term{overrides} all macro directives for the same name
that are visible at the point of the directive.
\item
\color{addclr}
A macro directive is \term{active} if it is visible and
no visible macro directive overrides it.
\item
\color{addclr}
A set of macro directives is \term{consistent} if it consists of only
\tcode{\#undef} directives or if all \tcode{\#define} directives in the set
are valid as redefinitions of the same macro.
\end{itemize}
\color{addclr}
When a \grammarterm{preprocessing-token} matching the macro name of a visible
macro directive is encountered, the set of active macro directives for that
macro name shall be consistent, and semantics of the active macro directives
determine whether the macro name is defined and the behavior of macro
replacement.
\enternote
The relative order of \grammarterm{pp-import}{s} has no bearing on whether a
particular macro definition is active.
\exitnote

\color{addclr}
\alinea
The \term{preprocessing preamble} of a translation unit is
the longest sequence of \grammarterm{preprocessing-token}{s} matching
the syntax of \grammarterm{pp-preamble}
beginning with the first \tcode{module} token that is
neither preceded by \tcode{extern} (\ref{dcl.module.proclaim})
nor followed by \tcode{;}.
If there is no such sequence of tokens,
the translation unit has no preprocessing preamble.
Within the preprocessing preamble, the tokens \tcode{export} and \tcode{import}
shall not be produced by macro replacement.
If an \tcode{import} \grammarterm{preprocessing-token}
appears after the end of the preprocessing preamble,
the program is ill-formed.
\enternote
The preamble always terminates with a semicolon token.
A \tcode{\#if} directive immediately following the end of the preamble
can expand macros imported by a \grammarterm{pp-import} within the
preamble; if that results in encountering additional
\grammarterm{preprocessing-token}{s} matching the syntax of a
\grammarterm{pp-import}, the preamble is not extended to include
those tokens because doing so would not form a valid preprocessing preamble.
\begin{example}
\begin{codeblock}
// macros.h
#define GOT_MACRO 1

// TU M
module M;
import "macros.h";
#if GOT_MACRO
import "other.h";
#endif
\end{codeblock}
The preprocessing preamble ends at the \tcode{;} of the first \tcode{import}.
As a consequence, the program is ill-formed because \tcode{import "other.h";}
appears after the preprocessing preamble.
The preprocessing preamble cannot extend to include the second \tcode{import},
because if it did, the macro \tcode{GOT_MACRO} would not be defined in the
context of the \tcode{\#if}.
\end{example}
\exitnote
\end{std.txt}

\end{after}
