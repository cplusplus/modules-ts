%!TEX root = ts.tex

\rSec0[lex]{Lexical conventions}

\setcounter{section}{0}
\Sec1[lex.separate]{Separate translation}

Modify paragraph 5.1/2 as follows
\begin{std.txt}
    \resetalinea[1]
    \alinea
    \enternote
    Previously translated translation units and instantiation units can be 
    preserved individually or in libraries. The separate translation units of 
    a program communicate (\ref{basic.link}) by (for example) calls to functions whose 
    identifiers have external \added{or module} linkage, manipulation of objects whose 
    identifiers have external \added{or module} linkage, or manipulation of data files. 
    Translation units can be separately translated and then later linked to produce 
    an executable program (\ref{basic.link}).  
    \exitnote

\end{std.txt}


\Sec1[lex.phases]{Phases of translation}

\noindent
Modify bullet 7 of paragraph 5.2/1 as follows:
\begin{std.txt}
%    \resetalinea[6]
%    \alinea
\begin{itemize}
    \item[7.]
    White-space characters separating tokens are no longer significant. 
    Each preprocessing token is converted into a token (5.6). 
    The resulting tokens are syntactically and semantically analyzed and 
    translated as a translation unit.
    \enternote
    The process of analyzing and translating the tokens may occasionally 
    result in one token being replaced by a sequence of other tokens (17.2). 
    \exitnote
    \added{It is implementation-defined whether the sources for}
    \begin{before}\added{module interface units for modules}\end{before}
    \begin{after}\added{module units}\end{after}
    \added{on which the current translation unit has an interface
    dependency (\ref{dcl.module.import}) are required to be available.}
    \enternote
    Source files, translation units and translated translation units need not
     necessarily be stored as files, nor need there be any one-to-one 
     correspondence between these entities and any external representation. 
     The description is conceptual only, and does not specify any particular 
     implementation.
    \exitnote 
\end{itemize}
\end{std.txt}

\begin{undecided}
Add new paragraphs as follows:
\begin{std.txt}\color{addclr}
    \resetalinea[1]
    \alinea
    The result of processing a translation unit from phases 1 through 7
    is a directed graph called the \term{abstract semantics graph} of
    the translation unit:
    \begin{itemize}
        \item Each vertex, called a \term{declset}, is
            a citation (\ref{dcl.module.import}), or a collection of 
            non-local declarations and redeclarations (Clause \ref{dcl.dcl})
            declaring the same entity or other non-local declarations
            of the same name that do not declare an entity.
        \item A directed edge $(D_1, D_2)$ exists in the graph if and only if the 
        declarations contained in $D_2$ declare an entity mentioned 
        in a declaration contained in $D_1$.
    \end{itemize}
    The \term{abstract semantics graph of a module} is
    \begin{before}\color{addclr}
    the subgraph of
    the abstract semantics graph of its module interface unit generated
    by the declsets the declarations of which are
    in the purview of that module interface unit.
    \end{before}
    \begin{after}\color{addclr}
    the subgraph of
    the abstract semantics graph of its module interface units generated
    by the declsets the declarations of which are
    in the purview of those module interface units.
    \end{after}
    \enternote
    The abstract semantics graphs of modules,
    as appropriately restricted (\ref{dcl.module.reach}), are used in
    the processing of \grammarterm{module-import-declaration}{s}
    (\ref{dcl.module.import}) and module implementation units.
    \exitnote

  \alinea
  An entity is \term{mentioned} in a declaration $D$ if that entity is a member of the
  \term{basis} of $D$, a set of entities determined as follows:
  \begin{itemize}
    \item If $D$ is a \grammarterm{namespace-definition}, the basis is the union
    of the bases of the \grammarterm{declaration}{s} in its
    \grammarterm{namespace-body}.

    \item If $D$ is a \grammarterm{nodeclspec-function-declaration},
    \begin{itemize}
        \item if $D$ declares a contructor, the basis is
        the union of the type-bases of the parameter types

        \item if $D$ declares a conversion function, the basis is
        the type-basis of the return type

        \item otherwise, the basis is empty.
    \end{itemize}

    \item If $D$ is a \grammarterm{function-definition}, the basis is
    the type-basis of the function's type

    \item If $D$ is a \grammarterm{simple-declaration}
    \begin{itemize}
        \item if $D$ declares a \grammarterm{typedef-name}, the basis is the
        type-basis of the aliased type
        \item if $D$ declares a variable, the basis is the type-basis
         of the type of that variable
         \item if $D$ declares a function, the basis is the type-basis of the type of that function
         \item if $D$ defines a class type, the basis is the union of the
         type-bases of its direct base classes (if any), and the bases of
         its \grammarterm{member-declaration}{s}.
         \item otherwise, the basis is the empty set.
    \end{itemize}

    \item If $D$ is a \grammarterm{template-declaration}, the basis is the union
    of the basis of its \grammarterm{declaration}, the set
    consisting of the entities (if any) designated by the default
    template template arguments, the default non-type template arguments, the
    type-bases of the default type template arguments.
    Furthermore, if $D$ declares a
    partial specialization, the basis also includes the primary template.

    \item If $D$ is  an \grammarterm{explicit-instantiation}
    or an \grammarterm{explicit-specialization}, the basis includes
    the primary template, and all the entities in the basis of the 
    \grammarterm{declaration} of $D$.

    \item If $D$ is a \grammarterm{linkage-specification}, the basis is the union
    of all the bases of the \grammarterm{declaration}{s} contained in $D$.

    \item If $D$ is a \grammarterm{namespace-alias-definition}, the basis is the 
    singleton consisting of the namespace denoted by the
    \grammarterm{qualified-namespace-specifier}.

    \item If $D$ is a \grammarterm{using-declaration}, the basis is the union
    of the bases of all the declarations introduced by
    the \grammarterm{using-declarator}.

    \item If $D$ is a \grammarterm{using-directive}, the basis is the
     singleton consisting of the norminated namespace.

    \item If $D$ is an \grammarterm{alias-declaration}, the basis is the type-basis of
    its \grammarterm{defining-type-id}.

    \item Otherwise, the basis is empty.
  \end{itemize}

  The \term{type-basis} of a type $T$ is
  \begin{itemize}
    \item If $T$ is a fundamental type, the type-basis is the empty set.
    \item If $T$ is a cv-qualified type, the type-basis is the type-basis of
    the unqualified type.
    \item If $T$ is a member of an unknown specialization, the type-basis
    is the type-basis of that specialization.
    \item If $T$ is a class template specialization, the type-basis
    is the union of the set consisting of the primary template and the
    template template arguments (if any) and the non-dependent non-type
    template arguments (if any), and the type-bases of the type 
    template arguments (if any).
    \item If $T$ is a class type or an enumeration type, 
    the type-basis is the singleton $\{ T \}$.
    \item If $T$ is a reference to $U$, or a pointer to $U$, or an array of $U$, the
    type-basis is the type-basis of $U$.
    \item If $T$ is a function type, the type-basis is the union of the
    type-basis of the return type and the type-bases of the parameter types.
    \item If $T$ is a pointer to data member of a class $X$, the type-basis is
    the union of the type-basis of $X$ and the type-basis of member type.
    \item If $T$ is a pointer to member function type of a class $X$, the
    type-basis is the union of the type-basis of $X$ and the type-basis of
    the function type.
    \item Otherwise, the type-basis is the empty set.
  \end{itemize}

%   The \term{expr-basis} of an expression $E$ is
%   \begin{itemize}
%     \item If $E$ is a pair of \grammarterm{expression} separated by a comma, 
%     the expr-basis is the union of the expr-bases of the two operands.

%     \item If $E$ is an assignment, the expr-basis is the union of the
%     expr-bases of the left operand and the right operand, and the set consisting 
%     of the function implementing the assignment operator (if the left operand
%     if of class type) and all the functions involved in the implicit conversion
%     sequences applied to the operands.
    
%     \item If $E$ is a \grammarterm{throw-expression}, the expr-basis is the
%     expr-basis of its operand.

%     \item If $E$ is a \grammarterm{conditional-expression}, the expr-basis is the
%     union of the expr-bases of all the operands, and the set of functions involved
%     in the implicit conversion sequences applied to the operands.

%     \item If $E$ is a binary expression, the expr-basis is the union of the 
%     expr-bases of the two operands, and the set of functions involved in the
%     implicit conversion sequences applied to the operands, and the function
%     selected by overload resolution (if any) for the operator.

%     \item If $E$ is an explicit type conversion expression, the expr-basis is
%     the union of the type-basis of the target type and the expr-basis of
%     the operand expression, and the set consisting of the function or constructor
%     involved in performing the conversion to the target type (if any).

%     \item If $E$ is a \grammarterm{noexcept-expression}, 
%     the expr-basis is the expr-basis of the operand.

%     \item If $E$ is an alignof expression,
%     the expr-basis is the type-basis of that operand.

%     \item If $E$ is a \grammarterm{delete-expression}, the expr-basis is
%     the union of the expr-basis of the operand and the set consisting of
%     the functions inolved in the implicit conversion sequence
%     applied to the operand, and the deallocation function selected.

%     \item If $E$ is a \grammarterm{new-expression}, the expr-basis is 
%     the union of
%     \begin{itemize}
%         \item the expr-bases of the expressions in its
%          \grammarterm{new-placement} (if any), and the set of
%          functions involved in the implicit conversion sequences applied
%          to those expressions (if any), and the allocation function selected
%          \item the type-basis of the \grammarterm{type-specifier-seq} contained
%          in the \grammarterm{new-type-id}, the expr-bases of the expressions
%          contained in the \grammarterm{new-declarator} (if any), and the set
%          consisting of the functions involved in the implicit conversion sequences
%          applied to those expressions
%          \item the expr-bases of the expressions contained in the
%          \grammarterm{new-initializer} (if any), and the set consisting of the
%          functions involved in the implicit conversion sequences applied 
%          to those expressions and the constructor selected (if any)
%     \end{itemize}

%     \item If $E$ is a sizeof expression, the expr-basis is the expr-basis
%      of the operand if the operand is an expression, otherwise the type-basis
%      of the operand.

%      \item If $E$ is prefix expression or a postfix expression, the expr-basis
%      is the expr-basis of the operand, and the set consisting of the 
%      operator function selected by overload resolution (if any).

%      \item If $E$ is a pointer to member expression, the expr-basis is the
%      set consisting of the designated non-static member.

%      \item If $E$ is any other unary expression, the expr-basis is the
%      union of the expr-basis of the operand expression, and the set
%      consisting of the operator function (if any) selected by overload
%      resolution and any functions involved in the implicit conversion 
%      sequence applied to the operand expression.

%      \item if $E$ is a \grammarterm{postfix-expression}
%      \begin{itemize}
%      \item and is a typeid expression, the expr-basis is the
%       type-basis of the type operand if the operand is a type, or the 
%       expr-basis of the expression operand

%       \item and is a cast expression, the expr-basis is the union of the
%       the type-basis of the target type, the expr-basis of the expression operand,
%       and the set consisting of function or constructor involved in the
%       conversion of the expression operand to the target type

%       \item and is a postifx increment or decrement expression, the expr-basis is
%       the expr-basis of the operand expression, and the set consisting of the
%       operator function (if any) selected by overload resolution

%       \item and is a pseudo destructor call, the expr-basis is the expr-basis
%       of the left-hand side of the dot or arrow operator

%       \item and is a class member access expression, the expr-basis is the union
%       of the left-hand expression operand  of the dot or the arrow operator,
%       and (if that left-hand side is not type-dependent) the set consisting of 
%       the class member designated by the \grammarterm{id-expression}, the 
%       operator function selected by overload resolution for the member access
%       operator

%       \item and is a function call expression, the expr-basis is the union of
%       the expr-basis of the \grammarterm{postfix-expression}, the expr-bases
%       of the arguments (if any), and the set consisting of the operator function
%       implementing the call selected by overload resolution, all the functions
%       involved in the implicit conversion sequences applied to each argument
%       in the call

%       \item and is a subscripting expression, the expr-basis is the union of
%       the expr-basis of the \grammarterm{postfix-expression}, and the expr-bases
%       of each expression in the \grammarterm{expr-or-braced-init-list}

%       \item otherwise, the expr-basis is the empty set.
%      \end{itemize}

%      \item If $E$ is a primay expression,
%      \begin{itemize}
%         \item and is a literal, the expr-basis is...
%         \item and is an \grammarterm{id-expression}, the expr-basis is the set
%         consisting of the designated entity (if it has namespace scope
%          or class scope) or the non-static class member; otherwise the class is empty
%          \item and is a parentesized expression, the expr-basis is the expr-basis
%          of that expression
%          \item and is a lambda expression, the expr-basis is the union of
%          \begin{itemize}
%             \item the type-bases in the parameter types (if any)
%             \item the type-basis of the \grammarterm{trailing-return-type} (if any)
%             \item the expr-basis of the \grammarterm{initializer}{s} (if any)
%             in its \grammarterm{lambda-capture}.
%          \end{itemize}
%      \end{itemize}

%      \item Otherwise, the expr-basis is the empty set.

%   \end{itemize}

%   The \term{expr-basis} of an expression $E$ is
%   \begin{itemize}
%    \item If $E$ is a \grammarterm{literal}
%    \begin{itemize}
%        \item the set consisting of the literal operator function called, if
%        $E$ is a \grammarterm{user-defined-literal}; otherwise
%        \item the empty set.
%    \end{itemize}

%    \item If $E$ is a parenthesized expression, the expr-basis is the expr-basis
%    of that expression.

%    \item If $E$ is a \grammarterm{id-expression}
%    \begin{itemize}
%        \item the set consisting of the designated destructor,
%         if $E$ denotes a destructor; otherwise,
%         \item the union of the set consisting of the template, the template
%         template-arguments (if any), the type-bases of the type
%         template-argumnts (if any), the expr-bases of the non-type 
%         template-arguments (if any), if $E$ is a \grammarterm{template-id}; otherwise,
%         \item the set consisting of the namespace scope or class scope entity
%         designated by $E$ if $E$ denotes such an entity; otherwise,
%         \item the empty set. 
%    \end{itemize}

%    \item If $E$ is a \grammarterm{lambda-expression}, the union of
%        \begin{itemize}
%            \item the type-bases in the parameter types (if any)
%            \item the type-basis of the \grammarterm{trailing-return-type} (if any)
%            \item the expr-basis of the expressions in the 
%            \grammarterm{initializer}{s} (if any)
%            in its \grammarterm{lambda-capture}.
%        \end{itemize}
    
%    \item If $E$ is a \grammarterm{fold-expression}, the union
%    of the expr-bases of the operands, and the set of operator functions
%    found by name lookup for the \grammarterm{fold-operator}{s} in the 
%    context of $E$. 

%    \item The empty set, if $E$ is any other \grammarterm{primary-expression}.

%    \item If $E$ is a subscripting expression, the union of the expr-basis
%    of the \grammarterm{postfix-expression}, the expr-bases of the
%    expressions contained in the \grammarterm{expr-or-braced-init-list}, and
%    the set consisting of the selected overloaded operator function (if any)
%    and all functions involved in the implicit conversion sequences (if any)
%    applied to those expressions.

%    \item If $E$ is a function call expression, the union of the expr-basis
%    of the \grammarterm{postfix-expression}, the expr-bases of the expressions
%    in the argument list, the set consisting of the selected overloaded function
%    operator (if any) and the functions involved in the 
%    implicit conversion sequences (if any) applied to the arguments,
%    and the set of functions found by name lookup in the context of $E$ if
%    the \grammarterm{postfix-expression} is a dependent name.

%    \item If $E$ is an explicit type conversion using functional notation,
%    the union of the type-basis of the target type, the expr-bases of the 
%    expressions in the \grammarterm{expression-list} or 
%    \grammarterm{brace-init-list}, the set consisting of the
%    constructor selected (if any) and all functions involved in
%    the implicit conversion sequences (if any) applied to the arguments.

%    \item If $E$ is a member selection expression, the union of the expr-basis
%    of the \grammarterm{postfix-expression} and the set consisting of
%    the overloaded function operator (if any) and the entity member (if any)
%    designated by the \grammarterm{id-expression}.

%    \item If $E$ is a named cast expression, the union of the type-basis
%    of the target type, the expr-basis of the operand expression, the 
%    set consisting of the function or constructor selected to perform
%    the conversion and the function involved in the implicit conversion
%    sequence (if any) applied to the operand.

%    \item If $E$ is a post-increment or post-decrement expression, the 
%    union of the expr-basis of the \grammarterm{postfix-expression}, and the
%    set consisting the selected function operator (if any).

%    \item If $E$ is a pseudo destructor call, the expr-basis of
%    the \grammartem{postfix-expression}.

%    \item If $E$ is a typeid expression, the expr-basis of the operand if
%     that operand is an expression, otherwise the type-basis of that operand.

%     \item The empty set, if $E$ is any other 
%     \grammarterm{postfix-expression}.

%     \item Otherwise, the empty set.
%\end{itemize}

\alinea
\enternote
The basis of a declaration includes neither non-fully evaluated expressions nor
entities used in those expressions.
\begin{example}
   \begin{codeblock}
    const int size = 2;
    int ary1[size];                         // \tcode{size} not in \tcode{ary1}'s basis
    constexpr int identity(int x) { return x; }
    int ary2[identity(2)];                  // \tcode{identity} not in \tcode{ary2}'s basis

    template<typename> struct S;
    template<typename, int> struct S2;
    constexpr int g(int);

    template<typename T, int N>
    S<S2<T, g(N)>> f();                     // \tcode{f}'s basis: $\{ \mathtt{S}, \mathtt{S2} \}$ 
   \end{codeblock}
\end{example}
\exitnote

\end{std.txt}
\end{undecided}

\begin{after}
\setcounter{section}{3}
\Sec1[lex.pptoken]{Preprocessing tokens}

Modify bullet 3 of paragraph 5.4/3 as follows:

\begin{std.txt}
Otherwise, the next preprocessing token is
the longest sequence of characters
that could constitute a preprocessing token,
even if that would cause further lexical analysis to fail,
except that a \grammarterm{header-name} (5.8)
is only formed
\added{when the previous preprocessing token
was lexically identical to the
\grammarterm{identifier} \tcode{import}, or}
within a \tcode{\#include} directive (19.2).
\end{std.txt}
\end{after}

\setcounter{section}{10}
\Sec1[lex.key]{Keywords}

In \ref{lex.key}, add these two keywords to Table 5 in paragraph
5.11/1: \added{module} and \added{import}.

\noindent
Modify note in paragraph \ref{lex.key}/1 as follows:
\begin{std.txt}
    \resetalinea[0]
    \alinea
    ...


    \enternote
    The \removed{\tcode{export} and} \tcode{register} keyword\removed{s are} \added{is}
    unused but \removed{are} \added{is} 
    reserved for future use.
    \exitnote
\end{std.txt}
