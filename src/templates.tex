%!TEX root = std.tex
\setcounter{chapter}{16}
\rSec0[temp]{Templates}%
%\indextext{template|(}

\noindent
Modify paragraph 17/2 as follows:
\begin{std.txt}
  \resetalinea[1]
  \alinea
  A \grammarterm{template-declaration} can appear only as a 
  namespace scope or class scope declaration.
  \added{Its \grammarterm{declaration} shall not be an
  \grammarterm{export-declaration}}
  \begin{before}\added{or a \grammarterm{proclaimed-ownership-declaration}}\end{before}
  \added{.}
  In a function template declaration, the last component of the 
  \grammarterm{declarator-id} shall not be a \grammarterm{template-id}. 
\end{std.txt}


%gram: \rSec1[gram.temp]{Templates}
%gram:

%% \setcounter{section}{5}
%% \rSec1[temp.res]{Name resolution}

%% \setcounter{subsection}{3}
%% \rSec2[temp.dep.res]{Dependent name resolution}

%% \noindent
%% Modify second bullet of paragraph 14.6.4/1
%% \begin{std.txt}
%%   \item[---] Declarations from namespace \added{partitions} associated
%%     with the types of the function arguments both from the
%%     instantiation context (14.6.4.1) and from the definition context. 
%% \end{std.txt}

\setcounter{section}{5}
\rSec1[temp.res]{Name resolution}
\setcounter{subsection}{3}
\rSec2[temp.dep.res]{Dependent name resolution}

Add new example to paragraph 17.6.4/1:
\begin{std.txt}
  \begin{addedblock}
  \enterexample
  \begin{codeblock}
    // Header file X.h
    namespace Q { 
       struct X { };
    }

    // Interface unit of M1
    @\begin{after}\color{addclr}
    module;
    \end{after}@
    #include "X.h"       // global module
    namespace Q {
       void g_impl(X, X);
    }
    export module M1;
    export template<typename T>
    void g(T t) {
       g_impl(t, Q::X{ });  // \#1: ADL in definition context finds Q::g_impl
    }

    // Interface unit of M2
    @\begin{after}\color{addclr}
    module;
    \end{after}@
    #include "X.h"
    import M1;
    export module M2;
    void h(Q::X x) {
       g(x);               // OK
    }
  \end{codeblock}
  \exitexample
\end{addedblock}
\end{std.txt}


\noindent
Add new paragraphs to 17.6.4:
\begin{std.txt}
   \color{addclr}
   \resetalinea[1]
   \alinea
   \enterexample
   \begin{codeblock}
     // Interface unit of Std
     export module Std;
     export template<typename Iter>
     void indirect_swap(Iter lhs, Iter rhs)
     {
        swap(*lhs, *rhs);    // swap can be found only via ADL
     }

     // Interface unit of M
     @\begin{after}\color{addclr}
     module;
     \end{after}@
     import Std;
     export module M;

     struct S { /* ...*/ };
     void swap(S&, S&);     // \#1;

     void f(S* p, S* q)
     {
        indirect_swap(p, q);   // instantiation finds \#1 via ADL
     }
   \end{codeblock}
   \exitexample
     
   \alinea
   \enterexample
   \begin{codeblock}
      // Header file X.h
      struct X { /* ... */ };
      X operator+(X, X);

      // Module interface unit of F
      export module F;
      export template<typename T>
      void f(T t) {
          t + t;
      }

      // Module interface unit of M
      @\begin{after}\color{addclr}
      module;
      \end{after}@
      #include "X.h"
      import F;
      export module M;
      void g(X x) {
         f(x);          // OK: instantiates f from F
                        // point of instantiation: just before g(X)
      }
   \end{codeblock}
   \exitexample

   \alinea
   \enternote
   \enterexample
   \begin{codeblock}
      // Module interface unit of A
      export module A;
      export template<typename T>
      void f(T t) {
         t + t;         // \#1
      }

      // Module interface unit of B
      export module B;
      import A;
      export template<typename T, typename U>
      void g(T t, U u) {
         f(t);
      }

      // Module interface unit of C
      @\begin{after}\color{addclr}
      module;
      \end{after}@
      #include <string>        // not in the purview of C
      import B;
      export module C;
      export template<typename T>
      void h(T t) {
         g(std::string{ }, t);
      }

      // Translation unit of main()
      import C;
      void i() {
         h(0);        // ill-formed: '+' not found at \#1
                      // point of instantiation of h<int>: just before 'i()'
                      // point of instantiation of g<std::string, int>: same as h<int>'s
                      // point of instantiation of f<std::string>: same as g<std::string, int>'s
      }
   \end{codeblock}
   \exitexample

   This example is ill-formed by this document.
   It is an open question as to how often the scenario occurs in
   practice, and whether to make the example well-formed or whether
   additional syntax will be introduced that does not involve
   modifying the header.
   \exitnote

   \alinea
   \enternote
   \enterexample
   \begin{codeblock}
    // Module interface unit of M1
    @\begin{after}\color{addclr}module;\end{after}@
    #include <algorithm>

    export module M1;
    export template<typename T, typename U>
    void f(T& t, U& u) {
      min(t, u);                          // \#1
    }

    // Module interface unit of M2
    @\begin{after}\color{addclr}module;\end{after}@
    #include <locale>
    struct Aux : std::ctype_base {
      operator int() const;
    };
    void min(Aux&, Aux&);   // \#2
    
    export module M2;
    import M1;
    export template<typename T>
    void g(T t) {
      Aux aux;
      f(aux, aux);
    }

    // Elsewhere, translation unit of global module
    import M2;
    void h() {
      g(0);
    }
   \end{codeblock}
   In the body of the function \tcode{h}, the call to \tcode{g} triggers a request
   for (implicit) instantiation of \tcode{g<int>}.  The point of instantiation of
   that specialization is right before the definition of \tcode{h}.
   That instantiation, in turn, requests the implicit instantiation of 
   \tcode{f<Aux,Aux>}.  The point of instantiation of that specialization
   immediately preceeds that of \tcode{g<int>}.  In that context, the invocation of
   \tcode{min}: (a) selects \tcode{std::min}; and (b) invokes the 
   implicit conversion.  In particular, the declaration at {\#2} is not used
   because it is neither available in the context of definition, nor in the
   context of instantiation of \tcode{f<Aux,Aux>}.
   However, paragraph 17.6.4.2/1 of the C++ Standard formally renders the behavior
   of the program undefined because the better match wasn't considered.  This is
   a case where it is unclear if that paragraph is too broad and needs further restrictions,
   or if there ought to be a mechanism to consider all such functions.
   \exitexample
   \exitnote
\end{std.txt}


\rSec3[temp.point]{Point of instantiation}

\noindent
Replace paragraph 17.6.4.1/7 as follows:
\begin{std.txt}
   \resetalinea[6]
   \alinea
   \removed{The instantiation context of an expression that depends on
   the template arguments is the set of declarations with external
   linkage declared prior to the point of instantiation of the
   template specialization in the same translation unit.}\added{The
   instantiation context of an expression that depends on template
   arguments is the context of a lookup at the point of instantiation
   of the enclosing template.}
\end{std.txt}


\rSec3[temp.dep.candidate]{Candidate functions}

\noindent
Modify paragraph 17.6.4.2/1 as follows
\begin{std.txt}
  \resetalinea[0]
  \alinea
  \ldots

  If the call would be ill-formed or would find a better match had the 
  lookup within the associated namespaces considered all the function 
  declarations with external   \added{or module}
  linkage introduced in those namespaces in all 
  translation units, not just considering those declarations found in the 
  template definition and template instantiation contexts, then the program 
  has undefined behavior.
\end{std.txt}



