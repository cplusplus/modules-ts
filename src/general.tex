
\rSec0[intro]{General}


\rSec1[intro.compliance]{Implementation compliance}

\pnum
Conformance requirements for this document are those 
defined in ISO 14882:2017, 4.1
except that references to the C++ Standard therein shall
be taken as referring to the document that is the result
of applying the editing instructions.  Similarly, all references
to the C++ Standard in the resulting document shall be taken
as referring to the resulting document itself.
\enternote 
Conformance is defined in terms of the behavior of programs.
\exitnote

%% %%
%% %% Feature-testing recommendations
%% %%
%% \rSec1[intro.features]{Feature-testing recommendations}

%% \pnum
%% An implementation that provides support for this Technical Specification shall 
%% define the feature test macro(s) in Table~\ref{tab:info.features}.

%% \renewcommand{\thetable}{\Alph{table}}
%% \begin{floattable}{Feature-test macro(s)}{tab:info.features}
%% {ll}
%% \topline
%% \lhdr{Macro name} & \rhdr{Value} \\
%% \capsep
%% \tcode{__cpp_concepts}  & \tcode{201507}      \\
%% \end{floattable}


% \pnum
% For the sake of improved portability between partial implementations of various
% C++ standards, WG21 (the ISO Technical Committee for the \Cpp Programming
% Language) recommends that implementers and programmers follow the guidelines in
% this section concerning feature-test macros. 
% \enternote 
% WG21's SD-6 makes similar recommendations for the \Cpp Standard.
% \exitnote

% \pnum
% Implementers who provide a new language feature should define a macro with the
% recommended name, in the same circumstances under which the feature is available
% (for example, taking into account relevant command-line options), to indicate
% the presence of support for that feature. Implementers should define that macro
% with the value specified in the most recent version of this Technical
% Specification that they have implemented. The macro name for this Technical 
% Specification is \tcode{__cpp_experimental_concepts}, and its value is 
% \tcode{201501}.

% \pnum
% No header files should be required to test macros describing the presence
% of support for language features.

\rSec1[intro.ack]{Acknowledgments}


\pnum
This document is based, in part, on the design and implementation
described in the paper P0142R0 ``\emph{A Module System for C++}''.
