%!TEX root = std.tex
\setcounter{chapter}{2}
\rSec0[diff]{Compatibility}%

\setcounter{section}{4}
\rSec1[diff.cpp17]{C++ and ISO C++ 2017}%

\setcounter{subsection}{0}
\rSec2[diff.cpp17.lex]{Clause 5: lexical conventions}%

\noindent
Add new entry as follows:
\begin{std.txt}
\color{addclr}
\textbf{Affected subclauses:} 5.8\\
\textbf{Change:} \grammarterm{header-name} tokens are formed in more contexts.\\
\textbf{Rationale:} Required for new features.\\
\textbf{Effect on original feature:}
When the identifier \tcode{import}
is followed by a \tcode{<} character,
a \grammarterm{header-name} token may be formed.
\begin{example}
\begin{codeblock}
template<typename> class import {};
import<int> f();                // ill-formed; previously well-formed
::import<int> g();              // OK
\end{codeblock}
\end{example}
\end{std.txt}

\setcounter{subsection}{2}
\rSec2[diff.cpp17.dcl]{Clause 9: declarations}%

\noindent
Add new entry as follows:
\begin{std.txt}
\color{addclr}
\textbf{Affected subclauses:} 9.11.1, 9.11.3\\
\textbf{Change:} New identifiers with special meaning.\\
\textbf{Rationale:} Required for new features.\\
\textbf{Effect on original feature:}
Top-level declarations beginning with
\tcode{module} or \tcode{import} may
either be ill-formed or interpreted differently
in this International Standard.
\begin{example}
\begin{codeblock}
class module;
module *m1;       // ill-formed; previously well-formed
::module *m2;     // OK

class import {};
import j1;        // was variable declaration; now \grammarterm{import-declaration}
::import j2;      // variable declaration
\end{codeblock}
\end{example}
\end{std.txt}
